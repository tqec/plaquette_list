\documentclass{article}

\usepackage[T1]{fontenc}
\usepackage[a4paper, margin=1in]{geometry}
\usepackage{graphicx} % Required for inserting images
\usepackage{subcaption}
\usepackage{pgffor}
\usepackage{underscore}
\usepackage{array}
\usepackage{longtable}
\usepackage{hyperref}

% https://tex.stackexchange.com/a/96468/166093
\usepackage[export]{adjustbox} % Max width/height in include graphics

\newcommand{\insertPlaquette}[1]{%
    \label{tabrow:#1} &%
    \includegraphics[width=\linewidth, max height=\linewidth]{plaquettes/pdf/#1.pdf} &%
    \includegraphics[width=\linewidth, max height=.09\paperheight]{circuits/pdf/#1.pdf}
}
% Easily adding plaquettes to the table
\newcommand{\internalTableLine}[3]{\texttt{#1} & \texttt{#2} & #3 & Row \ref{tabrow:#1}}

% Add ability to reference rows of tables
% See https://tex.stackexchange.com/a/356817/166093
\newcounter{rowcntr}[table]
\renewcommand{\therowcntr}{\thetable.\arabic{rowcntr}}
% A new columntype to apply automatic stepping
\newcolumntype{N}{>{\refstepcounter{rowcntr}\therowcntr}c}
% Reset the rowcntr counter at each new tabular
\AtBeginEnvironment{longtable}{\setcounter{rowcntr}{0}}

% Easily adding plaquette graphics to the table
%% New column, define the width as #1, the m-column will align all
%% content in ther vertical middle, \centering will align it in the
%% horizontal middle.
% See https://tex.stackexchange.com/a/345214/166093
% See also https://tex.stackexchange.com/a/12712/166093 for full explanations
\newcolumntype{C}[1]{>{\centering\let\newline\\\arraybackslash\hspace{0pt}}m{#1}}

\newcommand{\ket}[1]{\ensuremath{\left\vert #1 \right\rangle}}

\title{All plaquettes}
\author{Adrien Suau}
\date{November 2024}

\begin{document}

\maketitle

\section{Plaquette table}

\subsection{Explanation of prefixes, stabilizers and suffixes}

In this document, plaquettes are named according to specific scheme that is 
\texttt{<prefix><stabilizers><suffix>}. Each of these entries are explained in the next
sections.

\paragraph{Stabilizers} are the main part of each plaquette name. Only 4 values are currently used:

\begin{enumerate}
\item \texttt{XX} for 2-qubit \texttt{X} stabilizers,
\item \texttt{ZZ} for 2-qubit \texttt{Z} stabilizers,
\item \texttt{XXXX} for 4-qubit \texttt{X} stabilizers,
\item \texttt{ZZZZ} for 4-qubit \texttt{Z} stabilizers,
\end{enumerate}

\paragraph{Prefixes} are always located before the stabilizer string in any plaquette name.
The following values are allowed:

\begin{center}
\begin{tabular}{|l|p{.8\textwidth}|}
    \hline
    Prefix & Signification \\
    \hline 
    \texttt{<none>} & Regular memory plaquette \\
    \hline 
    \texttt{i} & Memory plaquette with data-qubit initialisation \\
    \hline 
    \texttt{m} & Memory plaquette with data-qubit measurement \\
    \hline 
    \texttt{mi} & Regular 4-qubit memory plaquette with the 2 data-qubit on the edge provided in \texttt{suffix} being initialised \\
    \hline 
    \texttt{mm} & Regular 4-qubit memory plaquette with 2 data-qubit on the edge provided in \texttt{suffix} being measured \\
    \hline 
    \texttt{th} & Regular memory plaquette with transversal $H$ gate \\
    \hline
\end{tabular}
\end{center}

\paragraph{Suffixes} are used to include a side of the plaquette in the overall name. Sides can
only be one of:

\begin{itemize}
\item \texttt{_UP}
\item \texttt{_DOWN}
\item \texttt{_LEFT}
\item \texttt{_RIGHT}
\end{itemize}

Depending on the values of \texttt{prefix} and \texttt{stabilizer}, the \texttt{suffix} 
may have different meanings:

\begin{itemize}
\item if \texttt{stabilizer} only contains 2 characters (e.g., \texttt{XX} or \texttt{ZZ})
    then the \texttt{suffix} contains the plaquette orientation (i.e., the side to which
    the rounded side is pointing to).
\item if \texttt{prefix} is \texttt{mi} or \texttt{mm} then the \texttt{suffix} contains the
    side on which qubits are initialized (for \texttt{mi}) or measured (for \texttt{mm}).
    Note that in that case \texttt{stabilizer} is always composed of 4 characters.
\end{itemize}


\subsection{List of all the plaquettes}
\begin{longtable}{ |l|l|l|l| } 
    % Main caption
    \caption{Exhaustive list of plaquettes that need to be implemented in \texttt{tqec}.\label{tab:exhaustivePlaquetteList}}\\
    % Headers
    %% First head
    \hline
    \multicolumn{4}{| c |}{Begin of Table}\\
    \hline
    Name & Category & Required for & Plaquette \& Circuit\\
    \hline
    \endfirsthead
    % Continuation head
    \hline
    \multicolumn{4}{| c |}{Continuation of Table \ref{tab:exhaustivePlaquetteList}}\\
    \hline
    Name & Category & Required for & Plaquette \& Circuit \\
    \hline
    \endhead

    % Footers
    %% Regular footer
    \hline
    \endfoot
    %% Last footer
    \hline
    \multicolumn{4}{| c |}{End of Table \ref{tab:exhaustivePlaquetteList}}\\
    \hline
    \endlastfoot

    % Actual table data
    \internalTableLine{XXXX}{Memory}{Memory} \\
    \internalTableLine{ZZZZ}{Memory}{Memory} \\
    \internalTableLine{XX_UP}{Memory}{Memory} \\
    \internalTableLine{XX_DOWN}{Memory}{Memory} \\
    \internalTableLine{XX_LEFT}{Memory}{Memory} \\
    \internalTableLine{XX_RIGHT}{Memory}{Memory} \\
    \internalTableLine{ZZ_UP}{Memory}{Memory} \\
    \internalTableLine{ZZ_DOWN}{Memory}{Memory} \\
    \internalTableLine{ZZ_LEFT}{Memory}{Memory} \\
    \internalTableLine{ZZ_RIGHT}{Memory}{Memory} \\
    \hline
    \internalTableLine{iXXXX}{Initialisation}{Memory} \\
    \internalTableLine{iZZZZ}{Initialisation}{Memory} \\
    \internalTableLine{iXX_UP}{Initialisation}{Memory} \\
    \internalTableLine{iXX_DOWN}{Initialisation}{Memory} \\
    \internalTableLine{iXX_LEFT}{Initialisation}{Memory} \\
    \internalTableLine{iXX_RIGHT}{Initialisation}{Memory} \\
    \internalTableLine{iZZ_UP}{Initialisation}{Memory} \\
    \internalTableLine{iZZ_DOWN}{Initialisation}{Memory} \\
    \internalTableLine{iZZ_LEFT}{Initialisation}{Memory} \\
    \internalTableLine{iZZ_RIGHT}{Initialisation}{Memory} \\
    \hline
    \internalTableLine{mXXXX}{Measurement}{Memory} \\
    \internalTableLine{mZZZZ}{Measurement}{Memory} \\
    \internalTableLine{mXX_UP}{Measurement}{Memory} \\
    \internalTableLine{mXX_DOWN}{Measurement}{Memory} \\
    \internalTableLine{mXX_LEFT}{Measurement}{Memory} \\
    \internalTableLine{mXX_RIGHT}{Measurement}{Memory} \\
    \internalTableLine{mZZ_UP}{Measurement}{Memory} \\
    \internalTableLine{mZZ_DOWN}{Measurement}{Memory} \\
    \internalTableLine{mZZ_LEFT}{Measurement}{Memory} \\
    \internalTableLine{mZZ_RIGHT}{Measurement}{Memory} \\
    \hline
    \internalTableLine{miXXXX_UP}{Memory \& Initialisation}{Merging} \\
    \internalTableLine{miXXXX_DOWN}{Memory \& Initialisation}{Merging} \\
    \internalTableLine{miXXXX_LEFT}{Memory \& Initialisation}{Merging} \\
    \internalTableLine{miXXXX_RIGHT}{Memory \& Initialisation}{Merging} \\
    \internalTableLine{miZZZZ_UP}{Memory \& Initialisation}{Merging} \\
    \internalTableLine{miZZZZ_DOWN}{Memory \& Initialisation}{Merging} \\
    \internalTableLine{miZZZZ_LEFT}{Memory \& Initialisation}{Merging} \\
    \internalTableLine{miZZZZ_RIGHT}{Memory \& Initialisation}{Merging} \\
    \hline
    \internalTableLine{mmXXXX_UP}{Memory \& Measurement}{Splitting} \\
    \internalTableLine{mmXXXX_DOWN}{Memory \& Measurement}{Splitting} \\
    \internalTableLine{mmXXXX_LEFT}{Memory \& Measurement}{Splitting} \\
    \internalTableLine{mmXXXX_RIGHT}{Memory \& Measurement}{Splitting} \\
    \internalTableLine{mmZZZZ_UP}{Memory \& Measurement}{Splitting} \\
    \internalTableLine{mmZZZZ_DOWN}{Memory \& Measurement}{Splitting} \\
    \internalTableLine{mmZZZZ_LEFT}{Memory \& Measurement}{Splitting} \\
    \internalTableLine{mmZZZZ_RIGHT}{Memory \& Measurement}{Splitting} \\
    \hline
    \internalTableLine{thXXXX}{Memory \& Transversal $H$}{Temporal transversal $H$} \\
    \internalTableLine{thZZZZ}{Memory \& Transversal $H$}{Temporal transversal $H$} \\
    \internalTableLine{thXX_UP}{Memory \& Transversal $H$}{Temporal transversal $H$} \\
    \internalTableLine{thXX_DOWN}{Memory \& Transversal $H$}{Temporal transversal $H$} \\
    \internalTableLine{thXX_LEFT}{Memory \& Transversal $H$}{Temporal transversal $H$} \\
    \internalTableLine{thXX_RIGHT}{Memory \& Transversal $H$}{Temporal transversal $H$} \\
    \internalTableLine{thZZ_UP}{Memory \& Transversal $H$}{Temporal transversal $H$} \\
    \internalTableLine{thZZ_DOWN}{Memory \& Transversal $H$}{Temporal transversal $H$} \\
    \internalTableLine{thZZ_LEFT}{Memory \& Transversal $H$}{Temporal transversal $H$} \\
    \internalTableLine{thZZ_RIGHT}{Memory \& Transversal $H$}{Temporal transversal $H$} \\
    \hline
\end{longtable}

\newpage
\section{Plaquette details}

\subsection{Memory}

The following plaquettes are used in the temporal bulk of a memory experiment. They are 
the simplest and most standard plaquettes you might already have seen in research papers.

\begin{longtable}{|N|C{.1\textwidth}|C{.3\textwidth}||N|C{.1\textwidth}|C{.3\textwidth}|} 
    % Main caption
    \caption{Visual representation and definition of \texttt{Memory} plaquettes that need to be implemented in \texttt{tqec}.\label{tab:memoryPlaquetteDefinitions}}\\
    % Headers
    %% First head
    \hline
    \multicolumn{6}{|c|}{Begin of Table}\\
    \hline
    \multicolumn{1}{|c|}{Index} & Visual representation & Definition (circuit) & \multicolumn{1}{c|}{Index} & Visual representation & Definition (circuit)\\
    \hline
    \endfirsthead
    %% Continuation head
    \hline
    \multicolumn{6}{|c|}{Continuation of Table \ref{tab:memoryPlaquetteDefinitions}}\\
    \hline
    \multicolumn{1}{|c|}{Index} & Visual representation & Definition (circuit) & \multicolumn{1}{c|}{Index} & Visual representation & Definition (circuit)\\
    \hline
    \endhead

    % Footers
    %% Regular footer
    \hline
    \endfoot
    %% Last footer
    \hline
    \multicolumn{6}{|c|}{End of Table \ref{tab:memoryPlaquetteDefinitions}}\\
    \hline
    \endlastfoot

    % Actual table data
    \insertPlaquette{XXXX} & \insertPlaquette{ZZZZ} \\
    \hline
    \insertPlaquette{XX_UP} & \insertPlaquette{ZZ_UP} \\
    \hline
    \insertPlaquette{XX_RIGHT} & \insertPlaquette{ZZ_RIGHT}\\
    \hline
    \insertPlaquette{XX_DOWN} & \insertPlaquette{ZZ_DOWN} \\
    \hline
    \insertPlaquette{XX_LEFT} & \insertPlaquette{ZZ_LEFT} \\
    \hline
\end{longtable}

\newpage
\subsection{Initialisation}

The following plaquettes are used in the first temporal round of a memory experiment. 
They are intializing the data-qubits in the $1$-eigenstate of the initialization basis 
$P \in \{ X, Y, Z \}$.
Below, $R_P$ is the operation that resets a single qubit to the $1$-eigenstate of $P$. That
definition gives $R_X \ket{\psi} \rightarrow \ket{+}$ and $R_Z \ket{\psi} \rightarrow \ket{0}$
for any $1$-qubit quantum state \ket{\psi}.

\begin{longtable}{|N|C{.1\textwidth}|C{.3\textwidth}||N|C{.1\textwidth}|C{.3\textwidth}|} 
    % Main caption
    \caption{Visual representation and definition of \texttt{Initialisation} plaquettes that need to be implemented in \texttt{tqec}.\label{tab:initialisationPlaquetteDefinitions}}\\
    % Headers
    %% First head
    \hline
    \multicolumn{6}{|c|}{Begin of Table}\\
    \hline
    \multicolumn{1}{|c|}{Index} & Visual representation & Definition (circuit) & \multicolumn{1}{c|}{Index} & Visual representation & Definition (circuit)\\
    \hline
    \endfirsthead
    %% Continuation head
    \hline
    \multicolumn{6}{|c|}{Continuation of Table \ref{tab:initialisationPlaquetteDefinitions}}\\
    \hline
    \multicolumn{1}{|c|}{Index} & Visual representation & Definition (circuit) & \multicolumn{1}{c|}{Index} & Visual representation & Definition (circuit)\\
    \hline
    \endhead

    % Footers
    %% Regular footer
    \hline
    \endfoot
    %% Last footer
    \hline
    \multicolumn{6}{|c|}{End of Table \ref{tab:initialisationPlaquetteDefinitions}}\\
    \hline
    \endlastfoot

    % Actual table data
    \insertPlaquette{iXXXX} & \insertPlaquette{iZZZZ} \\
    \hline
    \insertPlaquette{iXX_UP} & \insertPlaquette{iZZ_UP} \\
    \hline
    \insertPlaquette{iXX_RIGHT} & \insertPlaquette{iZZ_RIGHT}\\
    \hline
    \insertPlaquette{iXX_DOWN} & \insertPlaquette{iZZ_DOWN} \\
    \hline
    \insertPlaquette{iXX_LEFT} & \insertPlaquette{iZZ_LEFT} \\
    \hline
\end{longtable}

\newpage
\subsection{Measurement}

The following plaquettes are used in the last temporal round of a memory experiment. 
They are measuring the data-qubits in the measurement basis $P \in \{ X, Y, Z \}$.
Below, $M_P$ is the operation that measures a single qubit in the basis $P$.

\begin{longtable}{|N|C{.1\textwidth}|C{.3\textwidth}||N|C{.1\textwidth}|C{.3\textwidth}|} 
    % Main caption
    \caption{Visual representation and definition of \texttt{Measurement} plaquettes that need to be implemented in \texttt{tqec}.\label{tab:measurementPlaquetteDefinitions}}\\
    % Headers
    %% First head
    \hline
    \multicolumn{6}{|c|}{Begin of Table}\\
    \hline
    \multicolumn{1}{|c|}{Index} & Visual representation & Definition (circuit) & \multicolumn{1}{c|}{Index} & Visual representation & Definition (circuit)\\
    \hline
    \endfirsthead
    %% Continuation head
    \hline
    \multicolumn{6}{|c|}{Continuation of Table \ref{tab:measurementPlaquetteDefinitions}}\\
    \hline
    \multicolumn{1}{|c|}{Index} & Visual representation & Definition (circuit) & \multicolumn{1}{c|}{Index} & Visual representation & Definition (circuit)\\
    \hline
    \endhead

    % Footers
    %% Regular footer
    \hline
    \endfoot
    %% Last footer
    \hline
    \multicolumn{6}{|c|}{End of Table \ref{tab:measurementPlaquetteDefinitions}}\\
    \hline
    \endlastfoot

    % Actual table data
    \insertPlaquette{mXXXX} & \insertPlaquette{mZZZZ} \\
    \hline
    \insertPlaquette{mXX_UP} & \insertPlaquette{mZZ_UP} \\
    \hline
    \insertPlaquette{mXX_RIGHT} & \insertPlaquette{mZZ_RIGHT} \\
    \hline
    \insertPlaquette{mXX_DOWN} & \insertPlaquette{mZZ_DOWN} \\
    \hline
    \insertPlaquette{mXX_LEFT} & \insertPlaquette{mZZ_LEFT} \\
    \hline
\end{longtable}

\newpage
\subsection{Memory \& Initialisation}

The following plaquettes are used when 2 error-correction patches need to be merged 
together. They are ensuring that the data-qubits on the boundaries are correctly 
initialized.

\begin{longtable}{|N|C{.1\textwidth}|C{.3\textwidth}||N|C{.1\textwidth}|C{.3\textwidth}|} 
% \begin{longtable}{|c|c|} 
    % Main caption
    \caption{Visual representation and definition of \texttt{Memory \& Initialisation} plaquettes that need to be implemented in \texttt{tqec}.\label{tab:meminitPlaquetteDefinitions}}\\
    % Headers
    %% First head
    \hline
    \multicolumn{6}{|c|}{Begin of Table}\\
    \hline
    \multicolumn{1}{|c|}{Index} & Visual representation & Definition (circuit) & \multicolumn{1}{c|}{Index} & Visual representation & Definition (circuit)\\
    \hline
    \endfirsthead
    %% Continuation head
    \hline
    \multicolumn{6}{|c|}{Continuation of Table \ref{tab:meminitPlaquetteDefinitions}}\\
    \hline
    \multicolumn{1}{|c|}{Index} & Visual representation & Definition (circuit) & \multicolumn{1}{c|}{Index} & Visual representation & Definition (circuit)\\
    \hline
    \endhead

    % Footers
    %% Regular footer
    \hline
    \endfoot
    %% Last footer
    \hline
    \multicolumn{6}{|c|}{End of Table \ref{tab:meminitPlaquetteDefinitions}}\\
    \hline
    \endlastfoot

    % Actual table data
    \insertPlaquette{miXXXX_UP} & \insertPlaquette{miZZZZ_UP}\\
    \hline
    \insertPlaquette{miXXXX_RIGHT} & \insertPlaquette{miZZZZ_RIGHT}\\
    \hline
    \insertPlaquette{miXXXX_DOWN} & \insertPlaquette{miZZZZ_DOWN}\\
    \hline
    \insertPlaquette{miXXXX_LEFT} & \insertPlaquette{miZZZZ_LEFT}\\
    \hline
\end{longtable}

\newpage
\subsection{Memory \& Measurement}

The following plaquettes are used when 2 error-correction patches need to be split.
They are ensuring that the data-qubits on the boundaries are correctly measured.

\begin{longtable}{|N|C{.1\textwidth}|C{.3\textwidth}||N|C{.1\textwidth}|C{.3\textwidth}|} 
% \begin{longtable}{|c|c|} 
    % Main caption
    \caption{Visual representation and definition of \texttt{Memory \& Measurement} plaquettes that need to be implemented in \texttt{tqec}.\label{tab:memmeasPlaquetteDefinitions}}\\
    % Headers
    %% First head
    \hline
    \multicolumn{6}{|c|}{Begin of Table}\\
    \hline
    \multicolumn{1}{|c|}{Index} & Visual representation & Definition (circuit) & \multicolumn{1}{c|}{Index} & Visual representation & Definition (circuit)\\
    \hline
    \endfirsthead
    %% Continuation head
    \hline
    \multicolumn{6}{|c|}{Continuation of Table \ref{tab:memmeasPlaquetteDefinitions}}\\
    \hline
    \multicolumn{1}{|c|}{Index} & Visual representation & Definition (circuit) & \multicolumn{1}{c|}{Index} & Visual representation & Definition (circuit)\\
    \hline
    \endhead

    % Footers
    %% Regular footer
    \hline
    \endfoot
    %% Last footer
    \hline
    \multicolumn{6}{|c|}{End of Table \ref{tab:memmeasPlaquetteDefinitions}}\\
    \hline
    \endlastfoot

    % Actual table data
    \insertPlaquette{mmXXXX_UP} & \insertPlaquette{mmZZZZ_UP}\\
    \hline
    \insertPlaquette{mmXXXX_RIGHT} & \insertPlaquette{mmZZZZ_RIGHT}\\
    \hline
    \insertPlaquette{mmXXXX_DOWN} & \insertPlaquette{mmZZZZ_DOWN}\\
    \hline
    \insertPlaquette{mmXXXX_LEFT} & \insertPlaquette{mmZZZZ_LEFT}\\
    \hline
\end{longtable}

\newpage

\subsection{Temporal transversal Hadamard}

The following plaquettes are used the implement a temporal junction with a transversal 
Hadamard gate.

\begin{longtable}{|N|C{.1\textwidth}|C{.3\textwidth}||N|C{.1\textwidth}|C{.3\textwidth}|} 
    % Main caption
    \caption{Visual representation and definition of \texttt{Memory} plaquettes that need to be implemented in \texttt{tqec}.\label{tab:temporalTransversalHadamard}}\\
    % Headers
    %% First head
    \hline
    \multicolumn{6}{|c|}{Begin of Table}\\
    \hline
    \multicolumn{1}{|c|}{Index} & Visual representation & Definition (circuit) & \multicolumn{1}{c|}{Index} & Visual representation & Definition (circuit)\\
    \hline
    \endfirsthead
    %% Continuation head
    \hline
    \multicolumn{6}{|c|}{Continuation of Table \ref{tab:temporalTransversalHadamard}}\\
    \hline
    \multicolumn{1}{|c|}{Index} & Visual representation & Definition (circuit) & \multicolumn{1}{c|}{Index} & Visual representation & Definition (circuit)\\
    \hline
    \endhead

    % Footers
    %% Regular footer
    \hline
    \endfoot
    %% Last footer
    \hline
    \multicolumn{6}{|c|}{End of Table \ref{tab:temporalTransversalHadamard}}\\
    \hline
    \endlastfoot

    % Actual table data
    \insertPlaquette{thXXXX} & \insertPlaquette{thZZZZ} \\
    \hline
    \insertPlaquette{thXX_UP} & \insertPlaquette{thZZ_UP} \\
    \hline
    \insertPlaquette{thXX_RIGHT} & \insertPlaquette{thZZ_RIGHT}\\
    \hline
    \insertPlaquette{thXX_DOWN} & \insertPlaquette{thZZ_DOWN} \\
    \hline
    \insertPlaquette{thXX_LEFT} & \insertPlaquette{thZZ_LEFT} \\
    \hline
\end{longtable}

\end{document}
